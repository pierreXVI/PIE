\textbf{Résumé ---} L'intérêt du client se porte sur les méthodes de modélisation numériques des phénomènes physiques liés à la mécanique des fluides.

Les phénomènes concernés sont modélisés à l'aide d'équations aux dérivées partielles. Obtenir des solutions de celles-ci nécessite de résoudre à la fois un problème spatial et un problème temporel. Ce qui limite aujourd'hui les performances des codes de calcul est leur capacité à résoudre la partie temporelle du problème.

L'enjeu de ce projet était donc d'estimer le potentiel d'un nouveau type de méthodes temporelles, appelées méthodes exponentielles. Nous avons pu montrer qu'il y a un réel bénéfice à utiliser ces méthodes par rapport aux méthodes classiques sur certains types de problèmes. Pour d'autres types de problèmes, comme l'équation de Burgers, nous avons pu montrer que ces méthodes fournissent un résultat correct sans pouvoir conclure sur leurs performances réelles.

Ces méthodes nécessitent le calcul d'exponentielles de matrices pour fonctionner. Ce calcul étant lourd lorsque lesdites matrices sont de dimension conséquente, il est nécessaire d'utiliser des méthodes spécifiques pour accélérer les processus de calcul. Nous avons pu montrer qu'il était possible de réaliser un gain de temps substantiel en utilisant des méthodes basées sur les espaces de Krylov.

\textbf{Déclaration d'authenticité ---} Nous déclarons avoir personnellement réalisé ce projet, rédigé les différents documents et attestons également ne pas avoir eu recours au plagiat et avoir consciencieusement mentionné tout emprunt fait à autrui.


\noindent\rule[2pt]{\textwidth}{0.5pt}