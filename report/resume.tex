\vspace*{\fill}
\noindent\rule[2pt]{\textwidth}{0.5pt}\\
{\textbf{Résumé ---}}
L'intérêt du client porte sur les méthodes de modélisation numériques des phénomènes physiques liés à la mécanique des fluides. 

Les phénomènes physiques concernés sont modélisés à l'aide d'équations aux dérivées partielles. Obtenir des solutions de celles-ci nécessite de résoudre à la fois un problème spatial et un problème temporel. Aujourd'hui ce qui limite les performances des logiciels de CFD est leur capacité à résoudre la partie temporelle du problème. 

L'enjeu de ce projet était donc d'estimer le potentiel de nouvelles méthodes temporelles, appelées méthodes exponentielles. Nous avons pu montrer à travers ce projet qu'il y avait un réel bénéfice à utiliser les méthodes dites "Exponentielles" par rapport aux méthodes classiques. 
Néanmoins nous émettons des réserves quant à l'utilisation de ces méthodes sur des problèmes non linéaires de type équation de Burgers. %pourquoi? quels sont les résultats? d'où vient le problème. ça peut valoir le coup de le dire directement (Louis)

Comme leur nom l'indique, ces méthodes nécessitent le calcul d'exponentielles de matrices pour fonctionner. Ce calcul étant assez lourd lorsque lesdites matrices sont de dimension conséquente, il est nécessaire en pratique d'utiliser des méthodes spécifiques pour accélérer les processus de calcul.Nous avons pu montrer qu'il était possible de réaliser un gain de temps substantiel en utilisant des méthodes basées sur les espaces de Krylov.%montré qu'il est possible de l'accélérer significativement en utilisant les espaces de Krylov. 

{\textbf{Mots clefs :}}
Mots clefs
\\
\noindent\rule[2pt]{\textwidth}{0.5pt}